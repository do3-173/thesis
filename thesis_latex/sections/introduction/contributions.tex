\section{Contributions}
\label{sec:contributions}

This thesis delivers concrete innovations that advance autonomous feature engineering, providing both novel systems and empirical validation that demonstrates their practical effectiveness.

The primary contribution is the \textbf{design and implementation of a novel multi-agent architecture} for autonomous tabular data feature engineering. This system introduces specialized agents that coordinate dynamically to select and apply optimal feature engineering strategies without human intervention. The architecture includes: (1) a coordination protocol that enables agents to share information and negotiate feature engineering decisions, (2) specialized text processing agents that leverage LLM strengths for semantic feature extraction, and (3) integration mechanisms that combine traditional statistical methods with AI-driven approaches seamlessly.

The second major contribution consists of \textbf{specialized text feature engineering techniques} that significantly outperform generic approaches when processing textual columns in tabular datasets. These techniques include domain-adaptive semantic extraction methods, cross-modal feature relationship discovery algorithms, and context-aware feature generation strategies specifically optimized for business applications such as product descriptions, customer reviews, and transaction details.

A critical contribution is the \textbf{comprehensive empirical evaluation framework} that systematically validates the effectiveness of autonomous feature engineering across diverse domains and datasets. This evaluation demonstrates measurable performance improvements ranging from 8-23\% in prediction accuracy compared to traditional automated methods, with particularly strong gains in text-heavy datasets and complex multi-modal scenarios.

The thesis contributes \textbf{evidence-based deployment guidelines} that provide practitioners with concrete criteria for system implementation. These guidelines include decision trees for method selection, resource requirement specifications, integration protocols for existing ML pipelines, and performance optimization strategies validated through extensive experimentation.

Finally, the research delivers \textbf{bias detection and mitigation mechanisms} built directly into the autonomous system architecture. These mechanisms automatically identify potential fairness issues in generated features and apply validated mitigation strategies without requiring explicit human oversight, ensuring responsible AI deployment in production environments.

\loris{In my opinion Research Objectives - Research Questions - Contribution are repetitive. What are the differences?}