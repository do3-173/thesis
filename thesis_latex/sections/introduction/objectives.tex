\section{Research Objectives}
\label{sec:objectives}

This thesis aims to advance the understanding and practical application of Large Language Models for automated feature engineering in tabular data analysis. The research is structured around five interconnected objectives that collectively address experimental validation and practical implementation challenges.

The first major objective focuses on conducting \textbf{systematic experimental validation} of existing LLM-based feature engineering approaches across diverse tabular datasets. Rather than theoretical analysis, this objective emphasizes empirical evaluation to establish performance baselines, identify methodological strengths and limitations, and validate approaches under various data conditions. By systematically testing existing methods alongside novel approaches, this experimental foundation provides evidence-based guidance for method selection and identifies opportunities for improvement.

The second objective centers on \textbf{agentic AI system development}, specifically the design and implementation of novel multi-agent architectures for autonomous feature engineering on tabular data. This innovative approach introduces specialized agents for different data types—numerical, categorical, and text features—with intelligent coordination mechanisms that enable autonomous feature generation and validation. The agentic framework leverages LLMs' reasoning capabilities while incorporating traditional statistical methods through agent collaboration, creating a self-improving system that adapts to different dataset characteristics.

The third objective focuses on \textbf{text feature engineering specialization}, leveraging LLMs' natural language processing strengths to extract meaningful features from text columns in tabular datasets. This objective recognizes that while LLMs can handle all data types, they provide the greatest advantage when processing textual content such as product descriptions, customer reviews, medical notes, and transaction details. The research develops specialized techniques for semantic feature extraction, domain knowledge integration, and cross-type feature relationships that span text and other data types.

The fourth objective addresses \textbf{practical framework design} that bridges the gap between research prototypes and real-world deployment requirements. This involves developing systems that address genuine constraints faced by practitioners, including computational efficiency for large-scale deployment, interpretability requirements for regulated domains, and robust bias mitigation strategies that ensure fair and ethical feature generation. The framework must balance theoretical optimality with practical usability, providing clear guidelines for implementation and deployment.

\loris{Furthermore, this thesis aims also to employ LLM to do features engineering on text features. LLMs are originally designed to handle text, this characteristics can be combined to the reasoning capability to automatically design features that efficiently encode the text information.} \edo{ADDRESSED: Text feature engineering specialization is now the third main objective, emphasizing LLMs' natural language strengths for semantic feature extraction from textual columns.}

The fifth objective involves \textbf{comprehensive empirical evaluation} across diverse tabular datasets to validate the effectiveness of both the agentic AI system and specialized text feature engineering approaches. This evaluation assesses performance gains, robustness across different domains, and computational efficiency compared to traditional methods. The research includes ablation studies to identify critical components, sensitivity analysis for hyperparameter settings, and scalability testing across datasets of varying sizes and complexity levels.

The final objective is to develop \textbf{evidence-based implementation guidelines} that enable practitioners to effectively deploy the developed methods in production environments. These guidelines provide concrete recommendations on system architecture, resource requirements, integration strategies with existing ML pipelines, and decision criteria for when to employ agentic versus traditional approaches. The research emphasizes practical adoption by addressing real-world constraints such as computational budgets, latency requirements, and interpretability needs.

\loris{All the topic are really interesting, they are perfect for a thesis but they are slightly different to our work. The topics highlight 'literature review' more then a experimental work. What about LLMs for text based feature engineering? Reinforcement Learning by Machine Learning Feedback can wait for the moment.} \edo{ADDRESSED: Complete restructure implemented - shifted from literature review to experimental validation focus. Text-based feature engineering is now a core objective. RLMF deferred as suggested.}