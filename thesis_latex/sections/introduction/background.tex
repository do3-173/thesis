\section{Background and Motivation}
\label{sec:background}

Feature engineering represents one of the most critical yet labor-intensive components in machine learning workflows. Traditional approaches require extensive domain expertise, manual statistical analysis, and iterative experimentation to identify meaningful transformations that enhance model performance. This manual bottleneck significantly limits the scalability and accessibility of machine learning solutions, particularly for organizations lacking specialized data science expertise.

Recent advances in Large Language Models (LLMs), exemplified by GPT-4 and similar architectures, have demonstrated remarkable capabilities in automated reasoning, contextual understanding, and code generation. These capabilities present unprecedented opportunities for automating feature engineering tasks that traditionally required human expertise. Unlike computer vision or natural language processing domains where end-to-end deep learning has achieved remarkable success, tabular data analysis continues to depend heavily on carefully crafted feature transformations.

\textbf{Text-Based Feature Engineering Focus:} This thesis specifically emphasizes the application of LLMs to text-based feature engineering within tabular datasets. Text columns in tabular data—such as product descriptions, customer reviews, medical notes, or categorical labels—represent rich sources of semantic information that traditional statistical methods often fail to exploit effectively. LLMs' natural text processing capabilities, combined with their reasoning abilities, offer unique advantages for extracting meaningful features from such textual content.

The challenge lies in developing systematic approaches that leverage LLMs' semantic understanding while maintaining statistical rigor and computational efficiency. Current approaches often lack comprehensive experimental validation and fail to address practical deployment constraints such as computational cost, interpretability requirements, and bias mitigation.