\section{Foundations of LLM-Based Feature Engineering}
\label{sec:foundations}

The application of Large Language Models to tabular data analysis has emerged from the recognition that traditional feature engineering approaches, while statistically sound, often fail to leverage the rich semantic knowledge that humans naturally apply when understanding data \cite{li2024exploring}. Unlike computer vision or natural language processing domains where deep learning has achieved remarkable success through end-to-end learning, tabular data analysis has remained heavily dependent on manual feature engineering and domain expertise \cite{hollmann2024caafe}.

Recent advances in large language models, particularly with the development of models like GPT-4 \cite{openai2024gpt4} and their demonstrated few-shot learning capabilities, have opened new possibilities for automating feature engineering tasks. The fundamental premise underlying this research direction is that LLMs, trained on vast corpora of text that include descriptions of datasets, domain knowledge, and analytical procedures, can be prompted to generate meaningful feature transformations that would traditionally require human expertise.